\chapter{Files}\label{ch:ch4label}
\section{.WAV}

Waveform Audio File Format (WAVE, or more commonly known as WAV due to its filename extension)[3][6][7][8] (rarely, Audio for Windows)[9] is a Microsoft and IBM audio file format standard for storing an audio bitstream on PCs. It is an application of the Resource Interchange File Format (RIFF) bitstream format method for storing data in "chunks", and thus is also close to the 8SVX and the AIFF format used on Amiga and Macintosh computers, respectively. It is the main format used on Windows systems for raw and typically uncompressed audio. The usual bitstream encoding is the linear pulse-code modulation (LPCM) format.\\

Both WAVs and AIFFs are compatible with Windows, Macintosh, and Linux operating systems. The format takes into account some differences of the Intel CPU such as little-endian byte order. The RIFF format acts as a "wrapper" for various audio coding formats.\\
Though a WAV file can contain compressed audio, the most common WAV audio format is uncompressed audio in the linear pulse code modulation (LPCM) format. LPCM is also the standard audio coding format for audio CDs, which store two-channel LPCM audio sampled 44,100 times per second with 16 bits per sample. Since LPCM is uncompressed and retains all of the samples of an audio track, professional users or audio experts may use the WAV format with LPCM audio for maximum audio quality.[10] WAV files can also be edited and manipulated with relative ease using software.\\
The WAV format supports compressed audio, using, on Windows, the Audio Compression Manager. Any ACM codec can be used to compress a WAV file. The user interface (UI) for Audio Compression Manager may be accessed through various programs that use it, including Sound Recorder in some versions of Windows.\\
Beginning with Windows $2000$, a WAVE\_FORMAT\_EXTENSIBLE header was defined which specifies multiple audio channel data along with speaker positions, eliminates ambiguity regarding sample types and container sizes in the standard WAV format and supports defining custom extensions to the format chunk.[4][5][11]\\
There are some inconsistencies in the WAV format: for example, 8-bit data is unsigned while 16-bit data is signed, and many chunks duplicate information found in other chunks.\\

WAV info - \url{<https://en.wikipedia.org/wiki/WAV>}


