\chapter{Introduction}\label{ch:introduction}
The focus of this project will be on implementing a distributed speech recognition (DSR) system.
The main focus is to understand, design and implement an artificial neural network that can recognize human speech using deep learning methods.
The toolkit that we will use for the machine learning part was developed by the Google Brain Team and it`s now an open-source, software library, called TensorFlow (TF) \cite{tensorflow2015-whitepaper}.
At the time of this project, the newest update available is version 1.3.0.\\\\
Speech is probably the most important forms of communication.
It is the most natural and efficient way to communicate with
each other. Humans learn all the relevant skills during early
childhood, without any instruction, and they continue to rely
on speech communication throughout their life \cite{kamblespeech}.
Humans also want to have a similar, natural and efficient form of communication with machines. 
Until recently, the idea of holding a conversation with a computer seemed pure science fiction.\\\\
However, the situation is changing, and quickly.
A growing number of people now talk to their smartphones, asking them to set reminders, search for directions, or send email and text messages.
Chief technology officer of \copyright{} Nuance Communications, Vlad Sejnoha stated: "We`re at a transition point where voice and natural-language understanding are suddenly at the forefront" \cite{kamblespeech}.\\

\subsubsection{Reading guide}
In Chapter \ref{ch:speech_processing} the preprocessing part of the project is covered. In it, detailed explanations of historical and modern approaches are taken into consideration. Chapter \ref{ch:machine_learning} presents an introduction to the field of artificial neural networks and displays the learning process by gradually introducing more complex types of networks. Alongside the theoretical part, a practical introduction to TensorFlow is given in Chapter\ref{ch:machine_learning_platform}. Chapter \ref{ch:model_development} shows the difference between three working models, detailing the advantage of a bidirectional network model. Chapter \ref{ch:Network Framework} talks about the assembly and implementation of the distributed system. Chapter \ref{ch:implementation} and \ref{ch:discussion} provide a summary of the findings from the project, an analysis of future work and a closing argument. All the code and scripts used for the project can be found on the GitHub repository referenced here: \cite{mavericks2017}.
